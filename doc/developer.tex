\documentclass[12pt]{book}
\usepackage[T1]{fontenc}
\usepackage{longtable}
\usepackage{graphicx}
\usepackage{index}
\usepackage[colorlinks=true,hyperindex=true]{hyperref}
\makeindex

\begin{document}
\author{\href{http://freecol.sourceforge.net/index.php?section=8}{The FreeCol Team}}
\title{FreeCol Documentation\\Developer Guide for Version v0.9.0}
\maketitle{}

\tableofcontents
\newpage

\hypertarget{Changing the Rules}{\chapter{Changing the Rules}}

We would like to make FreeCol configurable, so that the game engine
becomes capable of emulating many similar games. For this purpose,
we have made many of the game's features configurable.

At some point in the future, we will probably add a special rule set
editor, but at the moment, your only option is to edit the file
specification.xml directly. This file defines the abilities of units,
founding fathers, buildings, terrain types, goods and equipment, for
example. You can find this file in the \textit{data/freecol} directory.

This is still work in progress, however, and the schema for the rule
set certain to change again in the future. If you wish to develop your
own rule set, you will have to monitor FreeCol development closely.

This having been said, we are particularly interested in hearing about
problems caused by your changes to the rule set. Some dialogs might be
unable to display more types of goods than are currently defined, for
example. Or other dialogs might not recognize your new Minuteman unit
as an armed unit. Please help us improve FreeCol by telling us about
such problems.

If you have a working rule set that adds a new flavour to the game, we
will gladly distribute it along with our default rule set. If you have
ideas that can not currently be implemented, we will probably try to
remove these limitations.

If you try to modify the rule set, you are strongly encouraged to
check whether the result is still valid. You can do this by validating
the result with the command \verb$ant validate$.


\hypertarget{Modifiers and Abilities}{\section{Modifiers and Abilities}}

Most of the objects defined by the rule set can be customized via
modifiers and abilities.  Abilities are boolean values (``true'' or
``false''). If the value is not explicitly stated, it defaults to
true. If an ability is not present, it defaults to false. Modifiers
define a bonus or penalty to be applied to a numeric value, such as
the number of goods produced by a unit. The modifier may be an
additive, multiplicative or a percentage modifier. Modifiers default
to ``identity'', which means they have no effect.

The code also checks that all abilities and modifiers it uses are
defined by the specification. Therefore, you must define all of them,
even if you do not use them. You can do this by setting their value to
the default value, e.g. ``false'' in the case of an ability, or ``0''
in the case of an additive modifier.

\newcommand{\ability}[1]{\index{#1}\index{Ability!#1}\hypertarget{#1}{\vspace{1em}\noindent\textbf{#1}}}
\newcommand{\modifier}[1]{\index{#1}\index{Modifier!#1}\hypertarget{#1}{\vspace{1em}\noindent\textbf{#1}}}
\newcommand{\affectsPlayer}{\\\textit{Affects: Player\\Provided by:
    Nation, Nation Type, Founding Father}}
\newcommand{\affectsUnit}{\\\textit{Affects: Unit\\Provided by:
    Nation, Nation Type, Founding Father, Unit Type, Equipment Type}}
\newcommand{\affectsBuilding}{\\\textit{Affects: Building\\Provided by:
    Building Type}}
\newcommand{\affectsColony}{\\\textit{Affects: Colony\\Provided by:
    Map}}
\newcommand{\affectsColonyTwo}{\\\textit{Affects: Colony\\Provided by:
    Building Type, Nation, Nation Type, Founding Father}}
\newcommand{\affectsTile}{\\\textit{Affects: Tile\\Provided by:
    Tile Type}}


\ability{model.ability.addTaxToBells}
\affectsPlayer

The player adds the current tax rate as a bonus to bells
production. The bonus is modified every time the tax increases or
decreases.

\ability{model.ability.alwaysOfferedPeace}
\affectsPlayer

The player is always offered peace in negotiations with AI players.

\ability{model.ability.ambushBonus}
\affectsUnit

The unit is granted an ambush bonus equal to the terrain's defence value.

\ability{model.ability.ambushPenalty}
\affectsUnit

The unit suffers an ambush penalty equal to the terrain's defence value.

\ability{model.ability.autoProduction}
\affectsBuilding

The building needs no units to produce goods, and will never produce
more goods than can be stored in the colony.

\ability{model.ability.automaticEquipment}
\affectsUnit

The unit automatically picks up equipment if attacked.

\ability{model.ability.automaticPromotion}
\affectsUnit

A unit that can be promoted will always be promoted when successful in
battle.

\ability{model.ability.betterForeignAffairsReport}
\affectsPlayer

The player is provided with more information about foreign powers.

\ability{model.ability.bombard}
\affectsUnit

The unit is able to bombard other units.

\ability{model.ability.bombardShips}
\affectsBuilding

The building has the ability to bombard enemy ships on adjacent tiles.

\ability{model.ability.bornInColony}
\affectsUnit

The unit can be born in a colony, provided that enough food is available.

\ability{model.ability.bornInIndianSettlement}
\affectsUnit

The unit can be born in an Indian settlement, provided that enough food is available.

\ability{model.ability.build}
\affectsBuilding

The building can build units or equipment.

\ability{model.ability.buildCustomHouse}
\affectsPlayer

The player can build custom houses.

\ability{model.ability.buildFactory}
\affectsPlayer

The player can build factories.

\ability{model.ability.canBeCaptured}
\affectsUnit

The unit can be captured. Land units that can not be captured are
destroyed, naval units that can not be captured are either sunk or
damaged.

\ability{model.ability.canBeEquipped}
\affectsUnit

The unit can be equipped.

\ability{model.ability.canNotRecruitUnit}
\affectsPlayer

The player can not recruit specified units.

\ability{model.ability.captureEquipment}
\affectsUnit

The unit can capture equipment from another unit.

\ability{model.ability.captureGoods}
\affectsUnit

The unit can capture goods from another unit.

\ability{model.ability.captureUnits}
\affectsUnit

The unit can capture enemy units.

\ability{model.ability.carryGoods}
\affectsUnit

The unit can transport goods.

\ability{model.ability.carryTreasure}
\affectsUnit

The unit can transport treasures, not treasure trains.

\ability{model.ability.carryUnits}
\affectsUnit

The unit can transport other units.

\ability{model.ability.convert}
\affectsUnit

The unit is a native convert.

\ability{model.ability.dressMissionary}
\affectsBuilding

The building can commission missionaries.

\ability{model.ability.electFoundingFather}
\affectsPlayer

The player can elect Founding Fathers.

\ability{model.ability.expertMissionary}
\affectsUnit

The unit is an expert missionary, but not necessarily commissioned.

\ability{model.ability.expertPioneer}
\affectsUnit

The unit is an expert pioneer, but not necessarily equipped with tools.

\ability{model.ability.expertScout}
\affectsUnit

The unit is an expert scout, but not necessarily equipped with horses.

\ability{model.ability.expertSoldier}
\affectsUnit

The unit is an expert soldier, but not necessarily equipped with muskets.

\ability{model.ability.expertsUseConnections}
\affectsPlayer

Experts working in factories can produce a small amount of goods even
if the raw materials are not available in the colony.

\ability{model.ability.export}
\affectsBuilding

The building can export goods to Europe directly.

\ability{model.ability.foundColony}
\affectsUnit

The unit can found new colonies.

\ability{model.ability.foundInLostCity}
\affectsUnit

The unit may be generated as the result of exploring a Lost City Rumour.

\ability{model.ability.hasPort}
\affectsColony

The colony has access to at least one water tile. This ability can not
be set by the specification, but it can be used as a required ability.

\ability{model.ability.ignoreEuropeanWars}
\affectsPlayer

The player will not be affected by the Monarch's declarations of war.

\ability{model.ability.improveTerrain}
\affectsUnit

The unit is able to improve terrain.

\ability{model.ability.independenceDeclared}
\affectsPlayer

The player has declared independence.

\ability{model.ability.mercenaryUnit}
\affectsUnit

The unit may be offered as a mercenary unit.

\ability{model.ability.missionary}
\affectsUnit

The unit is able to establish missions and incite unrest in native
settlements.

\ability{model.ability.moveToEurope}
\affectsTile

Units on the tile are able to move to Europe.

\ability{model.ability.multipleAttacks}
\affectsUnit

The unit can attack more than once.

\ability{model.ability.native}
\affectsUnit

The unit is a native unit.

\ability{model.ability.navalUnit}
\affectsUnit

The unit is a naval unit.

\ability{model.ability.pillageUnprotectedColony}
\affectsUnit

The unit is able to steal goods from and destroy buildings in an
unprotected colony.

\ability{model.ability.piracy}
\affectsUnit

The unit is a privateer.

\ability{model.ability.produceInWater}
\affectsBuilding

The building enables units to produce on water tiles.

\ability{model.ability.refUnit}
\affectsUnit

The unit can be part of the Royal Expeditionary Force.

\ability{model.ability.repairUnits}
\affectsBuilding

The building can repair units.

\ability{model.ability.royalExpeditionaryForce}
\affectsPlayer

The player is a Royal Expeditionary Force.

\ability{model.ability.rumoursAlwaysPositive}
\affectsPlayer

The player will always get positive results from exploring Lost City
Rumours.

\ability{model.ability.scoutForeignColony}
\affectsUnit

The unit can scout out foreign colonies.

\ability{model.ability.scoutIndianSettlement}
\affectsUnit

The unit can scout out native settlements.

\ability{model.ability.selectRecruit}
\affectsPlayer

The player can select a unit to recruit in Europe. This also applies
to units generated as a result of finding a Fountain of Youth.

\ability{model.ability.teach}
\affectsBuilding

The building enables experts to teach other units. However, the
building may place limits on the experience level of teachers.

\ability{model.ability.tradeWithForeignColonies}
\affectsPlayer

The player may trade goods in foreign colonies.

\ability{model.ability.undead}
\affectsUnit

The unit is an undead unit (used only in revenge mode).





\modifier{model.modifier.bombardBonus}
\affectsPlayer

The player's units are granted a bombard bonus when attacking.

\modifier{model.modifier.buildingPriceBonus}
\affectsPlayer

The player can build or buy buildings at a reduced price.

\modifier{model.modifier.defence}
\affectsUnit

The unit has a defence bonus or penalty.

\modifier{model.modifier.immigration}
\textit{Affects: Player\\Provided by: Goods Type}

Goods of this type contribute to the player's immigration points.

\modifier{model.modifier.landPaymentModifier}
\affectsPlayer

The player can buy Indian land at a reduced price.

\modifier{model.modifier.liberty}
\textit{Affects: Player\\Provided by: Goods Type}

Goods of this type contribute to the colony's and the owning player's
liberty points.

\modifier{model.modifier.lineOfSightBonus}
\affectsUnit

The unit has an increased line of sight.

\modifier{model.modifier.minimumColonySize}
\affectsColonyTwo

The population of the colony can not be voluntarily reduced below this
number. The modifier does not in any way affect a population reduction
due to starvation or other events.

\modifier{model.modifier.movementBonus}
\affectsUnit

The unit has an increased movement range.

\modifier{model.modifier.nativeAlarmModifier}
\affectsPlayer

The player generates less native alarm.

\modifier{model.modifier.nativeConvertBonus}
\affectsPlayer

The player has a greater chance of converting natives.

\modifier{model.modifier.nativeTreasureModifier}
\affectsPlayer

The player generates greater treasures when destroying native settlements.

\modifier{model.modifier.offence}
\affectsUnit

The unit has an offence bonus or penalty.

\modifier{model.modifier.religiousUnrestBonus}
\affectsPlayer

The player generates greater religious unrest in Europe.

\modifier{model.modifier.sailHighSeas}
\affectsUnit

The unit's travel time between Europe and the New World is reduced.

\modifier{model.modifier.tradeBonus}
\affectsPlayer

Prices in the player's market remain stable for longer.

\modifier{model.modifier.treasureTransportFee}
\affectsPlayer

The player pays a smaller fee for transporting treasures to Europe.

\modifier{model.modifier.warehouseStorage}
\affectsBuilding

The building increases the capacity of the warehouse.


\printindex

\end{document}
